\documentclass[11pt, a4paper]{article}

\usepackage[margin=1.5cm]{geometry}
\usepackage{setspace}
\usepackage[rgb]{xcolor}
\usepackage{verbatim}
\usepackage{subcaption}
\usepackage{amsgen,amsmath,amstext,amsbsy,amsopn,tikz,amssymb,tkz-linknodes}
\usepackage{fancyhdr}
\usepackage[colorlinks=true, urlcolor=blue,  linkcolor=blue, citecolor=blue]{hyperref}
\usepackage[colorinlistoftodos]{todonotes}
\usepackage{rotating}
\usepackage{booktabs}
\usepackage{paralist}
\usepackage{amsthm}

\pagestyle{fancy}
\lhead{\footnotesize Assignment 1: CS3243 Introduction to AI \\ \bf {Tutorial: T06  Group: A1 - 43}}
\rhead{\footnotesize Name: Julius Putra Tanu Setiaji  Matric: A0149787E \\ Name: Tan Zong Xian  Matric: A0167343A \\ Name: Tay Yu Jia  Matric: A0171240W}


\begin{document}
\section*{Solving 8-puzzle}
\section{Algorithm}
\subsection{A* search}
We implemented both programs using A* search but with different heuristic functions. Given that our heuristic functions are consistent, the graph-search version of A* search is known to be \textit{complete}, \textit{optimal} and \textit{optimally efficient}. Therefore, it would be able to solve the 8-puzzle (if solvable).

\section{Heuristics}
We decided on the following heuristic functions:
\begin{itemize}
	\item $h_1$: Manhattan distance
	\item $h_2$: Linear conflict
\end{itemize}

\subsection{Manhattan distance}
\textbf{Definition:} $h_1$ is the sum of the distances of the tiles from their goal positions. Since diagonal moves are not allowed, the distance is the sum of horizontal and vertical distances. 

\begin{proof}[Proof of consistency for $h_1$]
	$ $ \\
	For every node $n$, the step cost, $c(n, n')$, of getting to any of its successor node $n'$ is 1 because for each move, only one tile is allowed to move one step closer to or further away from its goal position, i.e. $h(n') = h(n) \pm 1$. When we substitute step cost as 1 into $h(n) \leq c(n, n') + h(n')$, we get $h(n) \leq 1 + h(n')$. It is then trivial to see that the condition holds, therefore $h_1$ is consistent.
\end{proof}

\subsection{Linear conflict}
\textbf{Definition:} Two tiles $t_j$ and $t_k$ are in a linear conflict if $t_j$ and $t_k$ are in the same line, the goal positions of $t_j$ and $t_k$ are both in that line, $t_j$ is to the right of $t_k$, and the goal position of $t_j$ is to the left of the goal position of $t_k$.\\

To derive the Linear Conflict estimate for any node $n$,
\begin{enumerate}
	\item Calculate the minimum number of tiles that must be removed from $row_1$ such that there are no more linear conflicts.
	\item Repeat Step 1 for the other rows and columns and sum them.
	\item $LinearConflict(n) = 2 * $ result from Step 2
\end{enumerate}

Using the above property, we augment the manhattan distance heuristic by adding the linear conflict estimate to it, thereby making it a more informed heuristic.

\begin{center}
$h_2(n)$ = $ManhattanDistance(n) + LinearConflict(n) = h_1(n) + LinearConflict(n)$
\end{center}

\begin{proof}[Proof of consistency for $h_2$]
	$ $\newline
	Let $md(n, x)$ be the manhattan distance of tile $x$ in node $n$ and $lc(n, r_i)$ be the number of tiles that must be removed from row $r_i$ to resolve linear conflicts. 
	Assume that tile $x$ is moving from row $r_{old}$ to $r_{new}$, while remaining in column $c_{old}$. The impact of the change in position of tile $x$ relative to its goal position can be split into three scenarios:
	\begin{enumerate}
		\item The goal position of $x$ is in neither row. $md(n', x) = md(n, x) \pm 1$. There are no new linear conflicts. Hence, $h_2(n') = h_2(n) \pm 1$ and $h_2(n') + c(n, n') = h_2(n') + 1 \geq h_2(n)$.
		\item The goal position of $x$ is in $r_{new}$. Since $x$ moved into the row containing its goal position, $md(n', x) = md(n, x) - 1$ and this may or may not have new linear conflicts in row $r_{new}$, so $lc(n', r_{new}) = lc(n, r_{new})$ or $lc(n', r_{new}) = lc(n, r_{new}) + 2$. Because $r_{old}$ is not the goal row of $x$, its presence would not have contributed to any linear conflict there so $lc(n', r_{old}) = lc(n, r_{old})$. Hence, $h_2(n') = h_2(n) \pm 1$ and $h_2(n') + c(n, n') = h_2(n') + 1 \geq h_2(n)$.
		\item The goal position of $x$ is in $r_{old}$. Since $x$ moved out of the row containing its goal position, $md(n', x) = md(n, x) + 1$ and we do not know whether it originally contributed to linear conflicts in $r_{old}$ so $lc(n', r_{old}) = lc(n, r_{old})$ or $lc(n', r_{old}) = lc(n, r_{old}) - 2$. Because $r_{new}$ is not the goal row of $x$, its presence would not have contributed to any linear conflict there so $lc(n', r_{new}) = lc(n, r_{new})$. Hence, $h_2(n') = h_2(n) \pm 1$ and $h_2(n') + c(n, n') = h_2(n') + 1 \geq h_2(n)$.
	\end{enumerate}
	In all three scenarios, $h_2$ remains consistent. By symmetry of the 8-puzzle, $h_2$ is similarly consistent for movements from column to column. 
\end{proof}

The above proof is adapted from a paper published by Hansson, Mayer and Yung in 1985 on \textit{Generating Admissible Heuristics by Criticizing Solutions to Relaxed Models}.
\section{Statistics}
For given inputs, report number of nodes generated, maximum size of frontier that is reached. Check that solution is within 300 moves.

\section{Analysis}
\end{document}
